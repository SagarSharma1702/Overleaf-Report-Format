In India, public transport plays a prominent part in every existent’s life. Buses are the extensively used public transport by Indian citizens to reach their destination in day-to-day life. As it's extensively used, there are numerous problems in the Indian bus system, like as no exact change, it's possible that both the passenger and conductor don't have change. In these cases, the conductor may not repay the balance to the passenger. The amount of paper needed to generate bus tickets is far too high as nearly all the passengers take tickets and just throw them down once they reach their destination. occasionally the passenger might lose the paper ticket, which results in buying the ticket again paying full fare.
The idea of this design is to change the current ticketing system of the bus into a digitalized and effective system through an android application to prevent the problems caused by it and to give a better trip for the passengers. This idea may help the nationals of India to go cashless, without having the problem of carrying change or taking it out in crowded buses.

\section{ Motivation behind project topic}
Awareness for green megalopolis has been a great motivating factor during this work. The existing manual system uses bulks of paper each time, getting one of the main reasons behind reduction of trees and timbers. On enforcing a digital pass system, we'd be capable to help the climate to some extent. This work also offers a beneficence in the digital India movement. It helps reduce paperwork, manual- work, and saves time. In this work, digital bus passes are generated using an android application which would help commuters on a day-to-day basis to issue passes, renew them and pay for it in a much more effective and easy way.
