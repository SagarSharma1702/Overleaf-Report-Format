\begin{itemize}
\item In the previous system, the conductor in the bus had to carry each passenger one by one.
\item The conductor further has to enquire each passenger about their destination and originate a ticket manually on a paper roll.
\item The conductor has to generate the ticket for the passenger to collect the bus fare.
\item The Passenger has to carry change for bus fare or the conductor has to replace the change, which frequently leads to conflict.
\item Even so, when checked again by the conductor, the passenger has to buy the ticket again paying the full bus fare if the hardcopy of the ticket is lost by the passenger.
\end{itemize}  
 All these points easily indicate that the bus ticket system isn't effective enough in terms of time operation, service and protection. Also using paper rolls for tickets isn't eco-friendly nowadays as there's scarceness of trees.
 The subsisting platforms and operations that are used to help commuters plan their trip uses mobile data for the connectivity and communication and GPS to get the real- time position of the bus( or other means of transport) relative to the commuter. There are results that offer a limited delicacy in metropolitan towns. Even so, these results aren't available to the other metropolises and also, they depend on factual data to give information. The Intelligent Transport System( ITS) s ground results can be studied to overcome these risks, would help the commuter to effectively use the public transport which includes lower waiting time. There are numerous executions of Intelligent Transport Systems around the world, each result designed to address a specific demographic region. There are existing results like tramTRACKER by Yarra Technologies in Melbourne, Australia and Google Maps is always there to feed the requirements of the metropolitan commuters. The factors of ITS Technologies are wireless communication like WiFi, WiMAX, RFID,etc. and computational technologies like AI, Real Time data processing, etc. The result is also approximately grounded on the Floating car data, which is based on the collection of place, velocity and direction of trip of the vehicles from mobile phones with a two- way communication of GPS enabled vehicles with Smartphone grounded applications. The task of calculating the shortest path of trip is computed with Dijkstra's Algorithm. Communication protocol used to develop this prototype uses 3G for internet connectivity with MQTT for communication passing from the bus to the commuters. MQTT allows the bus to communicate with the commuter directly.

\section{Study on existing ETA models}
ANNs are preferred results to solve nonlinear problems. Computing the estimated time of arrival of the bus( ETA) is one similar problem where ANNs are used. In this section the ANN infrastructures like MLP, MLR, major data grounded predictors are talked about. 
Over time numerous forecasting models for predicting traffic countries like trip time and traffic flow have been developed. The factual Data grounded forecasting model takes the supposition that the current traffic condition is stationary to forecast the bus arrival time based on the historical data of trips during the same time period. The results inferred that a reasonable forecasting of coming conditions is possible with factual standards at the same day of the week and time period of trip. This conclusion is empirical in our results as the day-to-day and weekly trip time were negligibly harmonious throughout the testing phase. Average trip Time models are grounded entirely on the literal average trip time. The average time is used directly or in combination with other inputs to estimate the appearance time of the bus. In utmost of these researches the proposed model outperformed the average time model. The average time model was shown to outperform some of the models like the multilinear retrogression models. In recent times, ANNs with the capability to solve complex, nonlinear problems are gaining popularity in estimating the appearance time of buses. An enhanced ANN model used back propagation to estimate the appearance time of the bus. An ANN model was developed that predicts the trip time of the bus using GPS data. This model was applied to study a bus route in Chennai, India. The ANN outperformed the standard Multiple Linear Regression( MLR).

\section{Study on existing Ticketing Models}
With a digital approach, we can remove the downsides of a traditional system. The extent of having a digital pass is vast as it's going to exclude the hassle of getting a physical pass on a day-to-day basis, and also at the same time, it reduces manual intervention keeping the system very much safe and secure. The commuters won't have to bother about bearing such a small denotation of money with themselves, rather they can fluently pay through their cards or their UPIs. This digital system, however, isn't only useful for the commuters, but it'll also be a great help to the bus authority as they will no longer suffer from any losses caused due to misuse of passes and thereby adding their profit. The passengers can also profit by the offers or discounts that the bus company may give sometimes over the digital platform. Digital pass systems for buses either use a website or an application to give services to commuters. For the development of this system, analysis of the traditional system is wanted. And to overcome its failings, a digital bus pass application is proposed with chromatic features similar as pass renewal, pass generation, payment, classification wise pass( pupil, women and senior, especially abled), editing of pass etc. Both bus authorities as well as the passengers can enjoy benefits of this digital approach. This newe-pass system isn't only advanced but also effective and effective. The idea of this approach is to automate the payment as well as pass issuing procedures with further safety and security than it was ahead. The passengers get the self-dependence of using any fare paying model that's allowing full self- administration. The primary crucial success factor of the digital pass is the interoperability and simplicity of the system from the passenger’s point- of- view. The passengers can be assured that they will be guided safely to use this epass application and get the maximum of it. This type of digital avenue exists in one of the European countries i.e., Germany and their success is truly huge and motivating. Countries like Germany are using e-ticketing for public transportation by associating with their public transport companies such as Verband Deutscher Verkehrsunternehmen( VDV) who established the VDV core application for bringing thee-ticketing approach in their country. With such a success, other countries have started to estimate the VDV core application. Australia has successfully developed a airman model for the same as well. As the time progresses, additional countries will be coming forward and will come apprehensive of the need of a digital approach in their transportation as well.
