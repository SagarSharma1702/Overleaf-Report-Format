S. Kazi, M. Bagasrawala, F. Shaikh and A. Sayyed[1] in the paper ”Smart E-Ticketing System for Public Transport Bus” discusses the problem in the modern public transport system and how their Application will overcome this problem. The application allows users to book bus tickets and allot themselves a seat if available. The application also provides the user with a list of buses for their route from that bus-stop. The list will also contain the information about seat availability and the expected time for the bus to reach that particular bus stop. 
\\

Saurabh C. and Balram T.[2] in the paper ”Public transport system ticketing system using RFID and ARM processor Perspective Mumbai bus facility B.E.S.T” discusses the ticketing and identification of the passenger in the public transport. This paper suggests building a RFID system using ARM processors that can identify passengers in public transport as well as all accounting purposes related to travelling expenses. Automated accounting of public transport can be used to provide useful estimates of the cost of travelling from one bus stop to another as well as the crowd density can be measured inside the public transport. But in the case of India measuring crowd density is of no use. Radio Frequency Identification (RFID) tags have been proposed to be used in this project. 
\\

S. Karthick. and A. Velmurugan [3] in the paper ”Android suburban railway ticketing with GPS as ticket checker” ex- plains how you can buy suburban tickets using just a smartphone application, where you can carry your ASR ticket in your smartphone as QR-code (Quick-Response). It uses the smartphone’s GPS facility to validate and delete your ticket automatically after a specific interval of time once the user reaches the destination. Ticket information is stored in a cloud database for security purposes which is missing in the present suburban system. On the other side, the ticket checker has a checker application to search and validate the user’s ticket information which has been stored in the cloud database.
\\

Shital Kotle, Korke Jayshree D., Kandharkar Snehal B., Gaikwad Pranali A. and Kale Geetanjali J. [4] in the paper ”Smart Bus Ticketing Destination Announcement System Using QR-Code” explains how an application is used to book bus tickets. The user is also able to book a ticket by application by selecting source and destination then QR code will be generated. In Conductor’s application, the conductor will scan the QR code generated on the passenger's application. User is able to see QR-Code, Travel route information.
