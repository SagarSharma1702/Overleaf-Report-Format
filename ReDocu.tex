
\begin{table}
\begin{center}
\begin{tabular}{|c|} \hline 
Report Documentation\\ \hline
Report code : CS-TE-Seminar 2022-23 \\		\hline
Report Number : 55 \\ 		\hline
Address (Details) : \\
SNJB’s Late Sau. Kantabai Bhavarlalji Jain College of Engineering,\\ Jain Gurukul, Neminagar, Chandwad, Nashik, Pin Code - 423 101, Maharashtra, India \\ \hline		
Author     : Sagar Ganesh Sharma \\
Address   : Ganesh Nagar, Near Satsang Bhavan, Manmad - 423 104, \\ Dist. : Nashik, Maharashtra, India \\
Email       : er.sagargsharma@gmail.com \\
Roll No.   : 55 \\
Mob. No. : 9960726868
\\ 	\hline	
Academic Year : 2022-2023 \\
Branch : Computer Engineering
\\ 		\hline

Key Words : Smart, Bus, System, Passenger, Conductor, Database, \\ Public transport, Bus tickets, Android application, QR Code, IoT, \\ Smart Bus, Smart Cities, MQTT, Estimated Time of Arrival (ETA), \\ Dynamic Shortest Path, Dijkstra’s Algorithm \\ 		\hline
\begin{tabular}{|p{3.3cm}|p{3cm}|p{3cm}|p{3cm}|p{3cm}|}
 Type of Report : FINAL & Report Checked By :    & Report Checked Date : & Guides Complete Name : 
Prof. Yogita K. Desai & Total Copies  : 2
\\\end{tabular} \\ \hline
\begin{tabular}{|p{15cm}|}
 Abstract : \\
Public transport is the cheapest and most reliable transportation system in India,\\ hence it has always been popular with the masses. Buses are an integral means of public trans- port which plays a vital role in transportation in India. The advancement in the transport system has been increasing in day-to- day life as more and more people rely on public transport to go to work, school, hospitals, etc. Even though the public transport buses have been providing fairly satisfactory services, there is a need for a smart and reliable system. The major problem in local buses are about issuing bus tickets, which often leads to conflict between the passenger and the conductor. Keeping this in mind we are developing an android application which will provide an efficient and smooth bus ticketing experience for both the user(passenger) and the service provider(conductor). The android application provides an interface for the bus ticketing system combined with the technology of QR Code for quick money transfer. QR Code or the Quick Response Code is the most dominant form of storing and exchanging information between devices. It’s a type of matrix BarCode and has more capacity than UPC Codes. Typically scanned and interpreted by a camera enabled smartphone, but also can be interpreted or generated by any camera device implemented with QR decoding logic. The passengers can go cashless using this application, and the conductor does not have to worry about returning change for the paid bus fare. By this application, we can minimise the usage of paper tickets which will also help in green India. 
\\\end{tabular} \\ \hline
  \end{tabular}
   \end{center}
\end{table}



