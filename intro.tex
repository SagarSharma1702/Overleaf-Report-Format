
\section{Introduction}
Bus service is an essential mode of transportation nowadays other than ever because of global warming as well as the state of the economy. 70 percent of India’s populations depend on buses to get to their destination on time. Due to the fast moving world, humans are in need of a smooth transport system. In metropolitan metropolises like Mumbai and Delhi, 10- 15 million people journey through public transport buses daily. As a large number of people board buses everyday it's frequently difficult for passengers to get the ticket and preserve it. Therefore this system, applying the advantages of technology will crack the problem of bus ticketing by digitising the procedure of money transfer for bus fare, ticket generation and storage journey details. It also makes it eco-friendly by eradicating the application of paper rolls for a charity towards Green IT and climate consciousness. This smart bus ticketing system will lift the online ticketing system to a new position by presenting QR code for the purpose of safe transaction of bus ticket fare.
\\

QR code (shortened from Quick Response code) is the brand for a type of two- dimensional barcode. QR Codes are machine- readable and the content inside them cannot be changed once generated and also provides a quick, easy, handy, precise and automatic data collection system. With the adding application and popularisation of wireless communication and portable devices technology, two- dimensional barcode technologies have been employed worldwide. QR Codes are generally scrutinised and interpreted by a camera enabled smartphone, but also can be interpreted or generated by any camera device enforced with QR decrypting sense. Transferring of money using QR code not only makes it secure but also easy to utilise by just scrutinising. It reduces the pain of entering the accurate amount as the generated QR Code will formerly have the information of the bus fare.
\\

India is home to over a billion people, designing effective transportation results for this great population is a huge challenge. A person from the North finds it tough to commute in the Southern metropolises and vice-versa. The main reason is the absence of information about the routes to reach the destinations and the frequentness of operation of the public transport. The waiting time can be minimised with real - time data processing that suggests alternative routes to support the commuter reach their destination quickly. There's a need for a real time transportation information system for the public. The information therefore attained by the commuter can prop up better trip planning and less waiting time for the buses, hence using public transport effectively. With the adding effectiveness of Intelligent Transport System results, the proposed prototype aims to give real time estimates of arrival time, line management, etc. This requires the information to be transmitted to the commuter directly from the bus and hence the MQTT protocol is applied. This prototype features a hardware implementation which acquires GPS data and computes the shortest path using Dijkstra’s Algorithm, transmits the information via MQTT protocol with an estimated time of appearance to the bus stops using ANNs. The real- time line management utility can enhance utilisation of buses more efficiently with data available from significant data on the traffic patterns and operating frequency demands. The major data acquired during each trip is used to make an average time grounded predictor. The historic data also provides a platform for analysis. The parameters like driving and traffic patterns, trip time, waits and exceptions can be inferred. This prototype is still in the early development stage.


